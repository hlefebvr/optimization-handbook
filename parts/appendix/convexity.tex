\chapter{Convex sets}

\section{Polyhedra}

\subsection{Definitions}

\subsubsection{Combinations and hulls}

First, the following definitions will allow us to introduce some geometrical notions :

\begin{definition}[Convex combination]
    Let $x_1,...,x_n$ be a finite set of vectors in a real vector space, a convex combination of these vectors is a vector of the form 
    \[
        \sum_{i=1}^n\alpha_ix_i\qquad\textrm{with } \sum_{i=1}^k\alpha_i = 1, \alpha\ge 0
    \]
\end{definition}

\begin{definition}[Convex hull]
    \[
        \conv{X} = \left\{ \sum_{i=1}^n\alpha_ix_i \middle| x_i\in X, \sum_{i=1}^n\alpha_i = 1, \alpha\ge 0 \right\}
    \]
\end{definition}

\begin{definition}[Conical combination]
    Let $x_1,...,x_n$ be a finite set of vectors in a real vector space, a conical combination of these vectors is a vector of the form 
    \[
        \sum_{i=1}^k \alpha_ix_i \qquad\textrm{with } \alpha\ge 0
    \]
\end{definition}

\begin{definition}[Conical hull]
\[
    \cone{X} = \left\{ \sum_{i=1}^n \alpha_ix_i \middle| x_i\in X, \alpha_i \ge 0 \right \}
\]
\end{definition}

\subsubsection{Polyhedra, polytopes and polyhedral cones}

\begin{definition}[Polyhedron] (plural : polyhedra)
    \[
        P = \{ x\in\R^n | Ax\le b \}
    \]
\end{definition}

\begin{definition}[Polytope] A polytope is the convex set a finite number of points. Alternatively, 
    \[
        P = \{ x\in\R^n | Ax\le b, u^- \le x \le u^+ \}
    \]
\end{definition}

\begin{definition}[Polyhedral cone]
    \[
        P = \{ x\in\R^n | Ax \le 0, x\ge 0 \}
    \]
\end{definition}

\begin{observation}[Important]
    Note that no real consenus has been reach on the definitions of polyhedra and polytopes. Sometimes, polyhedra corresponds to three dimensional solids with polygonial faces while polytopes denote the extension of a polyhedra to higher dimensions. 
\end{observation}

It comes from these definitions that every polytopes and every polyhedral cones are polyhedra. These geometrical objects are depicted in figure \ref{fig:polyhedral_geometry}. 

\begin{figure}[h!]
    \centering
    \begin{tikzpicture}
        \draw[<->] (7, 0) -| (0,13);
        \fill[fill=cyan!20] (0,0) -- (7, 3) -- (7, 6) -- cycle;
        \draw[thick] (0,0) -- (7,3);
        \draw[thick] (0,0) -- (7,6);
        \draw (5,3) node[fill=white] {Cone};

        \draw[pattern=grid] (1,5) -- (3, 4) -- (6,6) -- (7,8) -- (3,8) -- (1, 7) -- cycle;
        \draw (3,6) node[fill=white] {Polytope};

        \fill[fill=blue!20] (.5,13) -- (2, 10) -- (4, 9) -- (6,  10.3) -- (7,12) -- (7,13) -- cycle;
        \draw (.5,13) -- (2, 10) -- (4, 9) -- (6, 10.3) -- (7,12);
        \draw (4, 11) node[fill=white] {Polyhedron};
    \end{tikzpicture}
    \caption{A polyhdron, a polytope and a polyhedral cone depicted in 2D}
    \label{fig:polyhedral_geometry}
\end{figure}

\subsection{Theorems}

We first recall the following definition in order to properly enounce some important theorems. 

\begin{definition}[Minkowski sum]
    Let $A$ and $B$ be two vector spaces, the Minkowki sum is defined as
    \[
        A + B = \{ a + b | a\in A, b\in B \}
    \]
\end{definition}

\begin{theorem}[Affine Minkowski-Weyl]
    Let there be a polyhedron defined by a set of inequalities, $P = \{ x\in\R^n | Ax\le b \}$. There exists vectors $x_1,...,x_q\in\R^,n$ and $y_1,...,y_r\in\R^n$ such that
    \[
        P = \cone{x_1,...,x_q} + \conv{y_1,...,y_r}
    \]
\end{theorem}

\begin{theorem}[Decomposition theorem for polyhedra]
    \label{th:decomposition_polyhedra}
    A set $P$ of vectors in a Euclidean space is a polyhedron if and only if it is the Minkowki sum of a polytope $Q$ and a polyhedral cone $C$. 
    \[
        P = Q + C
    \]
\end{theorem}

\begin{figure}[h!]
    \centering
    \begin{tikzpicture}[scale=.75]
        \draw[<->] (10, 0) -| (0, 10);
        \fill[fill=cyan!20] (2.4,10) -- (0,0) -- (10,3.7) -- (10, 10);
        \draw[thick] (0,0) -- (2.4, 10);
        \draw[thick] (0,0) -- (10, 3.7);
        \fill[fill=blue!20, opacity=.6] (1,3) -- (3,1.5) -- (8, 3) -- (10,3.7) -- (10,10) -- (2.4,10) -- (1.5, 6) -- cycle;
        \draw[pattern=grid, opacity=.8] (1,3) -- (3,1.5) -- (8, 3) -- (5, 6) -- (1.5, 6) -- cycle;
        \draw (7,7) node[fill=white] { Polyhedron};
        \draw (1.5,1.5) node[fill=white] {Cone};
        \draw (4,4) node[fill=white] {Polytope};
    \end{tikzpicture}
    \caption{Illustration of theorem \ref{th:decomposition_polyhedra}}
    \label{fig:decomposition_polyhedra}
\end{figure}

This decomposition theorem is illustrated in figure \ref{fig:decomposition_polyhedra}. Intuitively, the polytope defines the \textit{lower} shape of the polyhedron and the polyhedral cone defines the \textit{queue} of the polyhedron. 
